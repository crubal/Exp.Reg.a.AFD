\documentclass[a4paper,10pt]{article}
\usepackage[utf8]{inputenc}

%opening
\title{}
\author{}

\begin{document}
\section{Manual de Usuario}
\subsection{Conversor de expresión regular a autómata finito determinista}
La terminal sera el punto de entrada para el usuario; sólo se tiene que ingresar la expresión regular 
la cual sera transformada a un Automata Finito Determinista. El programa sólo
reconoce símbolos individuales, es decir caracteres simples; éstos pueden ser
ingresados uno a continuación de otro o separados por espacios.
Luego, al costado de donde dice ER, se ingresa la expresión regular. Una
expresión regular esta conformada por los símbolos del alfabeto y los
operadores, estos son:
1. La disyunción +
2. La estrella de Kleen *
3. Los paréntesis ( )
Nótese que no se tiene en cuenta el operador de concatenación ya que este se omite.

El resultado del proceso de conversión se puede apreciar en una tabla de
transición de estados y posteriormente en un grafo dibujado mediante el lenguaje dot.

\end{document}
